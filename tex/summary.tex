% !TEX root = ../thesis.tex

\begin{summary}
图方法是机器学习、计算机视觉及数据挖掘等领域的一个传统的研究方法,而图结构中的相似性学习则贯穿了图方法的研究历史。数据驱动的相似性学习主要目的在于希望能够通过观测到的数据特征信息,在原有的启发式相似性的构造基础上,通过有目标的学习的方式,获取对于实际任务更优的顶点间连接或连接的相似性权重。通过相似性学习可以建立于实际任务目标相一致的图上相似性权重,在具有监督信息的情况下可以更进一步通过监督信息对其进行监督,从而有效地过滤通过固定规则建立的相似性权重中的噪声信息。
本文旨在针对实际的计算机视觉应用问题,研究其适用的相似性学习方法,并力图通过学习生成自适应的相似性权重使得整体算法在推断时更准确和鲁棒。我们在无监督图像聚类、半监督多模态约束聚类和监督的自然图像抠图三种计算机视觉应用问题上对相似性学习方法进行了研究,并大幅度地将所观测到的数据信息作用于相似性的学习上。

在本文中,我们首先从度量学习的角度研究了无监督下的相似性学习方法,将待学习的相似性矩阵假设为低秩的并进行矩阵分解,通过相似性矩阵与图嵌入交替优化的方式,在单一的局部性优化目标下实现相似性学习。相似性矩阵的低秩假设和对图嵌入的线性投影近似使得我们所提出的相似性学习方法具有极高的学习效率。然后本文在多模态约束聚类问题上对约束传播方法中的相似性学习进行了研究,同时指出了传统的无监督多模态相似性矩阵融合的方式本身具有一定的不稳定性。我们进一步就如何在多模态约束聚类问题中充分利用约束标签及相似性矩阵自身的一致性,学习出对目标问题更有的相似性矩阵进行了研究,并基于此提出了两个多模态约束聚类场景下的统一相似性矩阵生成方法。自然图像抠图作为一个关注图像局部相似度的应用,传统抠图方法多依赖于图方法进行学习。因此,我们对深度学习框架下如何在自然图像抠图任务中进行有效的相似性学习,并同时在有效性和高效性两个方面分别提出了层次化不透明度传播方法和归纳引导滤波器方法。

数据驱动的相似性学习在方法上还有很多值得研究的方向,例如:

(1)更高效的相似性学习方法:目前较为高效的相似性学习方法多数基于给定的图结构,借助已知的稀疏边实现相似性学习。该类方法无法超越已有的稀疏连接结构去探索未出现的边。然而对未知边的探索往往需要对所有潜在可能性进行遍历,具有极高的复杂度。所以,如何通过采样等灵活的方式,仅对高概率的潜在边进行考察并对相似性进行学习和预测,是一个有趣的研究方向。

(2)更有效的相似性学习监督方法:多数基于神经网络的相似性学习方法利用经过信息传播或图卷积后的特征数据间接,根据其传播后的推理效果简洁对相似性的学习进行监督。越来越多的方法直接对网络中间的相似性矩阵进行监督\cite{yu2020context,wang2020affinity}。因此如何解决大规模数据中的相似性标签不平衡问题,同时有效地实现对相似性矩阵地直接监督也是一个值得研究的方向。

(3)传统方法思路与神经网络参数化的更深入的结合:AdaAM方法中所采用的低秩相似性矩阵的假设则与Non-local网络\cite{wang2018non}等部分自注意力机制中的相似性矩阵构造方式具有较大的共同点。同时ILMCP方法中对相容条件概率的极大似然估计也可以通过网络进行参数化,改进为有监督的归纳学习。因此充分利用神经网络的学习能力与数据优势更进一步发掘传统算法设计的优势,一直是一个值得不断研究的方向。
\end{summary}
